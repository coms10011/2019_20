
\ifind
\section*{Example question}
\else
\subsection*{7 continuous distribution - example question}
\fi
The Kumaraswamy distribution is a continuous probability density with
support on $[0,1]$; this means its probability density is zero except
when $x\in [0,1]$. It has two parameters $a$ and $b$, both positive
and the probability density is
\begin{equation}
p(x)=abx^{a-1}(1-x^a)^{b-1}
\end{equation}
For $a=b=1$ it reduces to the uniform distribution. Its mean and
variance are surprisingly tricky to calculate, but its cumulative isn't. Calculate its cumulative.

\noindent \textbf{solution} Well for $x\in[0,1]$ the cumulative is
\begin{equation}
F(x)=\int_0^x p(y)dy=\int_0^x aby^{a-1}(1-y^a)^{b-1}dy 
\end{equation}
and this can be done using the substitution $u=1-y^a$ so that $du=-ay^{a-1}$ giving
\begin{equation}
F(x)=-\int_1^{1-x^a} bu^{b-1}du=-u^b|_1^{1-x^a}=1-(1-x^a)^b 
\end{equation}
