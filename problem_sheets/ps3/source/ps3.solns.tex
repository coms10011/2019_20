%ps3.tex
%notes for the course Probability and Statistics COMS10011 
%taught at the University of Bristol
%2018_19 Conor Houghton conor.houghton@bristol.ac.uk

%To the extent possible under law, the author has dedicated all copyright 
%and related and neighboring rights to these notes to the public domain 
%worldwide. These notes are distributed without any warranty. 

\documentclass[11pt,a4paper]{scrartcl}
\typearea{12}
\usepackage{graphicx}
%\usepackage{pstricks}
\usepackage{listings}
\usepackage{color}
\lstset{language=C}
\usepackage{fancyhdr}
\pagestyle{fancy}
\lfoot{\texttt{github.com/COMS10011/2018\_19}}
\lhead{COMS10007 ps3 - Conor / Anne - outline solutions}
\begin{document}

\section*{Problem Sheet 3}



\subsection*{Questions}

\begin{enumerate}

\item The size of a standard croquet ball is 3 5/8
  inches\footnote{Everything in croquet is measured in old timey
    units}. The height of a croquet hoop is 3 3/4 inches. If a not
  very good croquet-ball making machine makes croquet balls whose mean
  matches the standard and with standard deviation 1/8 inch, what is
  the chance it will make a ball too large to fit through the
  hoop. You can write the solution in terms of the error function.
  \\ \\ \\ \textbf{Solution}: So
\begin{equation}
z=\frac{x-\mu}{\sqrt{2}\sigma}
\end{equation}
so for $x_1=3.75$ in, we have
\begin{equation}
z_1=\frac{1/8}{\sqrt{2}/8}=\frac{1}{\sqrt{2}}
\end{equation}
Any height bigger than this will not fit, so $z_2=\infty$ and $\mbox{erf}{\infty}=1$ so
\begin{equation}
\mbox{Prob}(x>3.75)=\frac{1}{2}[1-\mbox{erf}(1/\sqrt{2})]\approx 0.16
\end{equation}
where the 0.16 is given for interest, it wasn't expected as part of the answer.

\item This will look like a long question but it is almost all
  background and the question is not too bad when you actually read
  through it. In particle physics when a collider is being used to
  find a new particle like the Higgs boson or the top squark
  scientists don't detect the sought after particle directly since it
  usually decays almost straight away, instead they detect the more
  common particles that particle will decay into, for example, a Higgs
  boson can decay in to two photons and these can be detected. Roughly
  speaking scientists count these events. However, the whole situation
  is very messy and there will always be some events even if the
  particle doesn't exist at the energy being examined. The amount of
  these background events will fluctuate from experiment to
  experiment, typically like a Gau\ss{}ian. The scientific team is
  allowed to claim they have discovered the particle if the number of
  events they measure is more than five standard deviations above
  what would be expected if the particle didn't exist. What is the
  probability of this \lq{}discovery\rq{} happening by chance?
  \\ \\ \\ \textbf{Solution}: So we are interested in the probability of a results bigger than $\mu+5\sigma$. Now
\begin{equation}
z_1=\frac{\mu+5\sigma-\mu}{\sqrt{2}\sigma}=\frac{5}{\sqrt{2}}
\end{equation}
and
\begin{equation}
\mbox{Prob}(x>\mu+5\sigma)=\frac{1}{2}[1-\mbox{erf}(5/\sqrt(2))]
\end{equation}
which is about one chance in 3.5 million.

\end{enumerate}

\subsection*{Extra question}

\begin{itemize}
\item Another useful distribution is the exponential distribution:
$$
p(x)=\left\{\begin{array}{cc}\lambda e^{-\lambda x}& x\ge 0\\ 0&\mbox{otherwise}\end{array}\right.
$$
What is the probability $\mbox{Prob}(x_1 < x <x_2)$ where $x_1$ and $x_2$ are both positive.
  \\ \\ \\ \textbf{Solution}: 
\begin{equation}
\mbox{Prob}(x_1 < x <x_2)=\lambda \int_{x_1}^{x_2} e^{-\lambda y}dy=\left. e^{-\lambda y} \right]_{x_1}^{x_2}=e^{-\lambda x_1}-e^{-\lambda x_2}
\end{equation}
\end{itemize}



\subsection*{Extra questions}

\begin{itemize}

\item The Dirac delta function $\delta(x)$ is a special function that is zero everywhere but when $x=0$, but
$$
    \int_{-\infty}^\infty\delta(x)dx=1
    $$
Integrating by parts or similar shows that this means
$$
    \int_{-\infty}^\infty\delta(x)f(x)dx=f(0)
    $$
You can think of it as the $a\rightarrow 0$ limit of the uniform distribution with support $[-a,a]$. If the random variable $X$ has $p(x)=\delta(x-1)$ it means $X=1$. If $Y$ is uniformely distributed from zero to two, what is the distribution for $Z=X+Y$?
\\ \\ \\ \textbf{Solution}: So we know
$$p_Z(z)=\int_{-\infty}\infty \delta(x-1)p_Y(z-x)dx=p_Y(z-1)$$
so $Z$ is uniformly distributed over the interval $[1,3]$.

\item Australorp hens weigh on average 4kg with a standard deviation
  0.25kg; australorps who weight less than 3.5kg are fed
  \textsl{patent chicken spicer}, a mixture of chalk, corn and
  pepper. What fraction of these hens are fed patent chicken spicer?
  \\ \\ \\ \textbf{Solution}: Here is a picture of one of my hens, an Australorp, being fed a raspberry.
  \begin{center}
    \includegraphics[width=6cm]{hen.jpg}
  \end{center}
  So we are asking for $P(x<3.5)$ and since
  $$z=\frac{3.5-4}{0.25\sqrt{2}}=\sqrt{2}$$
    the probability, and hence the proportion, is
    $$P(x<3.5)=\frac{1}{2}[1-\mbox{erf}\,(\sqrt{2})]\approx 0.02275$$
  
\end{itemize}



\end{document}

