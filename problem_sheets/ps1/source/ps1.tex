%ps1.tex
%notes for the course Probability and Statistics COMS10011 
%taught at the University of Bristol
%2019_20 Conor Houghton conor.houghton@bristol.ac.uk

%To the extent possible under law, the author has dedicated all copyright 
%and related and neighboring rights to these notes to the public domain 
%worldwide. These notes are distributed without any warranty. 

\documentclass[11pt,a4paper]{scrartcl}
\typearea{12}
\usepackage{graphicx}
%\usepackage{pstricks}
\usepackage{listings}
\usepackage{color}
\lstset{language=C}
\usepackage{fancyhdr}
\pagestyle{fancy}
\lfoot{\texttt{coms10011.github.io}}
\lhead{COMS10011 ps1 - Conor}
\begin{document}

\section*{Problem Sheet 1}

\begin{center}
\includegraphics[width=7.5cm]{frequentists_vs_bayesians.png}
\end{center}

This is an xkcd cartoon that someone posted to the unit reddit last year, it is
\texttt{https://xkcd.com/1132/}. It is also worth reading Randall Monroe's comment on the response to this cartoon:

\begin{quote}
Hey! I was kinda blindsided by the response to this comic.

I�m in the middle of reading a series of books about forecasting
errors (including Nate Silver�s book, which I really enjoyed), and
again and again kept hitting examples of mistakes caused by blind
application of the textbook confidence interval approach.

Someone asked me to explain it in simple terms, but I realized that in
the common examples used to illustrate this sort of error, like the
cancer screening/drug test false positive ones, the correct result is
surprising or unintuitive. So I came up with the sun-explosion
example, to illustrate a case where na�ve application of that
significance test can give a result that�s obviously nonsense.

I seem to have stepped on a hornet�s nest, though, by adding
�Frequentist� and �Bayesian� titles to the panels. This came as a
surprise to me, in part because I actually added them as an
afterthought, along with the final punchline. (I originally had the
guy on the right making some other cross-panel comment, but I thought
the �bet� thing was cuter.)

The truth is, I genuinely didn�t realize Frequentists and Bayesians
were actual camps of people�all of whom are now emailing me. I thought
they were loosely-applied labels�perhaps just labels appropriated by
the books I had happened to read recently�for the standard textbook
approach we learned in science class versus an approach which more
carefully incorporates the ideas of prior probabilities.

I meant this as a jab at the kind of shoddy misapplications of
statistics I keep running into in things like cancer screening (which
is an emotionally wrenching subject full of poorly-applied
probability) and political forecasting. I wasn�t intending to
characterize the merits of the two sides of what turns out to be a
much more involved and ongoing academic debate than I realized.

A sincere thank you for the gentle corrections; I�ve taken them to
heart, and you can be confident I will avoid such mischaracterizations
in the future!

At least, 95.45\% confident.
\end{quote}
Thanks again to the person who posted this.

\subsection*{Useful facts}

\begin{itemize}

\item \textbf{Combinations} The number of ways of choosing $r$ items out of $n$ is 
\begin{equation}
\left(\begin{array}{c}n\\r\end{array}\right)=\frac{n(n-1)(n-2)\ldots(n-r+1)}{r(r-1)(r-2)\ldots 1}=\frac{n!}{r!(n-r)!} 
\end{equation}


\item \textbf{Combinations} The number of ways of splitting $n$ items into sets of size $r_1$, $r_2$ through to $r_k$ with
  \begin{equation}
    r_1+r_2+\ldots+r_k=n
  \end{equation}
  is
\begin{equation}
\left(\begin{array}{c}n\\r_1,r_2,\ldots,r_k\end{array}\right)=\frac{n!}{r_1!r_2!\ldots r_k!} 
\end{equation}


\item \textbf{Bayes's rule}
\begin{equation}
P(A|B)=\frac{P(B|A)P(A)}{P(B)}
\end{equation}

\item \textbf{Set notation}:
  \begin{itemize}
  \item $A\cup B$ is the union so $A\cup B={x|x\in A\mbox{ or }x\in B}$. If $A=\{1,2,3\}$ and $B=\{3,4,5\}$ then $A\cup B=\{1,2,3,4,5\}$
  \item $A\cap B$ is the intersection so $A\cap B={x|x\in A\mbox{ and }x\in B}$. If $A=\{1,2,3\}$ and $B=\{3,4,5\}$ then $A\cup B=\{3\}$
   \item $A\setminus B$ is the set minus so $A\setminus B={x|x\in A\mbox{ and }x\not\in B}$. If $A=\{1,2,3,4\}$ and $B=\{1,3,5\}$ then $A\setminus B=\{2,4\}$
   \item If $C$ is a subset, the complement of $C$, that is the set of all the elements not in $C$, is written $\bar{C}$. If $X=\{1,2,3,4\}$ and $C=\{1,2\}$ then $\bar{C}=\{3,4\}$.
  \end{itemize}
  For events, $A\cup B$ is the event of $A$ or $B$ happening, $A\cap
  B$ is the event of $A$ and $B$ happening, $A\setminus B$ is the
  event of $A$ happening but $B$ not happening and $\bar{C}$ is the
  event of $C$ not happening.
 
\item \textbf{Cards}: 52 cards made up of four suits; in each suit
  there are 13 values, ace, two through to ten and the jack, queen,
  kind.

\item \textbf{Poker hands}: the number of poker hands is 
\begin{equation}
\left(\begin{array}{c}52\\5\end{array}\right)= 2598960
\end{equation}

\end{itemize}


\subsection*{Questions}

Five questions each worth two marks and two marks for attendance but
with a maximum of ten marks.

\begin{enumerate}

\item In the poker hand \textsl{two pair} there are two pairs of cards
  with each card in the pair matched by value; the fifth card has a
  different value to either pair. What is the probability of two pairs
  when five cards are drawn randomly.


\item In a \textsl{full house} there is one
  pair and one triple, what is the probability of getting a full
  house?

\item A student answers a multiple choice question with four options,
  one of which is correct. 80\% of students know the answer, 20\% of
  students guess and choose randomly. If a student gets the answer
  correct what is the chance they knew the answer.

\item In the xkcd cartoon above, what is the chance the Bayesian will
  win his or her bet if the chance the sun has exploded is one in a
  million? In reality the chance is, of course, much less than one in
  a million! Show the answer to six decimal places.

\item A three-sided dice is rolled three times. $X$ is the sum of the
  largest two values. Write down the probability distribution for $X$.

\end{enumerate}

\subsection*{Extra questions}

These are for you to do on your own, not for handing up. Solutions will be included in the solutions section.

\begin{enumerate}

\item When it started in 1987 the Irish lottery has 36 numbers;
  participants paid 50 Irish pence to buy a combination of six
  different numbers; they would win if these numbers matched the six
  drawn. In the last week in May in 1992 a syndicate tried to buy all
  combinations of numbers. If they had succeeded how many numbers
  would they have bought? In the event the lottery shut down lots of
  the lottery machines so they only bought most of the numbers, they
  nonetheless had the winning number but shared the prize three
  ways. However, because of the roll-over prize and the match-5 and
  match-4 prizes, they are thought to have made a substantial
  profit. The lottery was redesigned after this to have more numbers.


\item From a group of three undergraduates and five graduate students,
  rour students are randomly selected to act as TAs. What is the
  chance there will be exactly two undergraduate TAs?

\item Prove
\begin{equation}
\left(\begin{array}{c}n\\r\end{array}\right)=\left(\begin{array}{c}n\\n-r\end{array}\right)
\end{equation}

\item Two events $A$ and $B$ have probabilities $P(A)=0.2$, $P(B)=0.3$ and $P(A\cup B)=0.4$. Find
\begin{enumerate}
\item Find $P(A\cap B)$.
\item Find $P(\bar{A}\cap \bar{B})$.
\item Find $P(A|B)$.
\end{enumerate}

\item One night in a bar in Las Vegas you meet a dodgy character who
  tells you that there are two types of slot machine in the Topicana,
  one that pays out 10\% of the time, the other 20\%. One sort of
  machine is blue, the other red. Unfortunately the dodgy character is
  too drunk to remember which is which. The next day you randomly
  select red to try, you find a red machine and put in a coin. You
  lose. Assuming the dodgy character was telling the truth, what is
  the chance the red machine is the one that pays out more. If you had
  won instead of losing, what would the chance be?\footnote{I stole
    this problem from \texttt{courses.smp.uq.edu.au/MATH3104/}}


\end{enumerate}

\end{document}

