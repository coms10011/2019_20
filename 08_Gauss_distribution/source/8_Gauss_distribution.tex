%8_Gauss_distribution
%notes for the course Probability and Statistics COMS10011
%taught at the University of Bristol
%Conor Houghton conor.houghton@bristol.ac.uk
%To the extent possible under law, the author has dedicated all copyright
%and related and neighboring rights to these notes to the public domain 
%worldwide. These notes are distributed without any warranty. 

\documentclass[11pt,a4paper]{scrartcl}
\typearea{12}
\usepackage{graphicx}
%\usepackage{pstricks}
\usepackage{listings}
\usepackage{color}


\newif\ifind
\indtrue


\lstset{language=C}
\usepackage{fancyhdr}
\pagestyle{fancy}
\lfoot{\texttt{github.com/COMS10011/2018\_19}}
\lhead{COMS100011 9\_Gauss\_distribution - Conor}
\begin{document}

\section*{8 The Gau\ss{}ian distribtion}

Recall how continuous probability distributions work, they are defined
in terms of a density function $p(x)$ where $p(x)$ is like the
probability per length, so 
\begin{equation}
\mbox{Prob}(x_1<x<x2)=\int_{x_1}^{x_2} p(x)dx
\end{equation}
We used $f(x)$ for the probability density when we introduced it to
avoid confusing it with the probabilities that are used for discrete
random variables. However, like almost everything in statistics it is
usually just called $p(x)$, or sometimes $p_X(x)$ if there are a few
random variables around and we want to know which probability density
goes with which variable.

The Gau\ss\footnote{\ss{} is a German letter equivalent to ss} or Gauss
or normal or Gau\ss{}ian or Gaussian or bell-curve distribution is a
continuous distribution which is used to model a whole range of
natural phenomenon, in fact, much of statistics and almost all
statistics outside of science, assumes almost everything has a
Gau\ss{}ian distribution. We will see why later on, basically there is
a theorem, the Central Limit Theorem, that tells us why the
Gau\ss{}ian distribution is as common as it is. For now though we will
look at the distribution and its properties.

\begin{figure}[tb]
\begin{center}
\include{fig_gauss}
\end{center}
\caption{A normal curve with mean zero and variance one, that is $\mu=0$ and $\sigma^2=1$.\label{fig_gauss}}
\end{figure}

The Gau\ss{}ian distribution is given by
\begin{equation}
p(x)=\frac{1}{\sqrt{2\pi\sigma^2}}e^{-\frac{(x-\mu)^2}{2\sigma^2}}
\end{equation}
It has a classic \lq{}bell\rq{} shape seen in Fig.~\ref{fig_gauss}; it
looks a bit like a binomial distribution with $p=0.5$. The slightly
confusing thing is the $1/\sqrt{2\pi\sigma^2}$, that is there to normalize the curve:
\begin{equation}
\int_{-\infty}^\infty e^{-x^2/2\sigma^2}dx=\sqrt{2\pi\sigma^2}
\end{equation}
This is confusing because we can do this particular definite integral
going from minus infinity to infinity, but the corresponding
indefinite integral can't be done in the sense that we can't write
down a formula in terms of a set of functions we already know. There is a trick
for doing the definite integral from minus infinity to infinity which
we won't look at here for reasons of time but is very nice if you want
to look it up. Of course, since the integral goes from minus infinity
to infinity shifting the $x$ around doesn't change the value:
\begin{equation}
\int_{-\infty}^\infty e^{-(x-\mu)^2/2\sigma^2}dx=\sqrt{2\pi\sigma^2}
\end{equation}
where $\mu$ is a constant. We will see that the mean is $\mu$ and the variance
$\sigma^2$, just as we'd hope, that's why these particular symbols
were used in the formula for the density.

The Gau\ss{}ian distribution is sometimes described as
$\mathcal{N}(\mu,\sigma^2)$; this notation is a little confusing, it is never
really specified what \lq{}described as\rq{} means, but roughly
speaking people write $X\sim \mathcal{N}(\mu,\sigma^2)$ as a shorthand for say
$X$ is normally distributed with mean $\mu$ and variance $\sigma^2$

\subsection*{The mean}

As we noted above, obviously when the constants were named $\mu$ and
$\sigma^2$ in the definition of the probability density it was because
these contants correspond to the mean and variance. We will check this
by working out the mean; this could be done with an integration by
parts, but it is more elegant to follow the proceedure we used for
calculating the mean for the binomial distribution.
\begin{equation}
1=Z=\frac{1}{\sqrt{2\pi \sigma^2}}\int_{-\infty}^\infty e^{-(\mu-x)^2/2\sigma^2}dx
\end{equation}
so
\begin{equation}
0=\frac{dZ}{d\mu}=-\frac{1}{\sqrt{2\pi \sigma^2}}\int_{-\infty}^\infty\frac{\mu-x}{2\sigma^2} e^{-(\mu-x)^2/2\sigma^2}dx
\end{equation}
and hence
\begin{equation}
\frac{1}{\sqrt{2\pi \sigma^2}}\int_{-\infty}^\infty xe^{-(\mu-x)^2/2\sigma^2}dx=\mu\frac{1}{\sqrt{2\pi \sigma^2}}\int_{-\infty}^\infty e^{-(\mu-x)^2/2\sigma^2}dx=\mu
\end{equation}
The left hand side is $\langle x\rangle$ and so we are done. A similar
approach, either taking the second derivative of $Z$ with respect to
$\mu$, or the first derivative with respect to $\sigma$ gives the
variance.


\subsection*{Working out Gau\ss{}ian probabilities}


\begin{figure}[tb]
\begin{center}
\include{fig_prob}
\end{center}
\caption{Working out the probability means calculating the area under the curve.\label{fig_prob}}
\end{figure}


\begin{figure}[tb]
\begin{center}
\include{fig_erf}
\end{center}
\caption{The error function $\mbox{erf}(x)$.\label{fig_erf}}
\end{figure}


Obviously what we'd like to do is work out probabilities:
\begin{equation}
\mbox{Prob}(x_1<x<x_2)=\frac{1}{\sqrt{2\pi\sigma^2}}\int_{x_1}^{x_2} e^{-\frac{(x-\mu)^2}{2\sigma^2}}dx
\end{equation}
as illustrated in Fig.~\ref{fig_prob}. The problem is that we can't
do that integral, there is no way to write the integral in terms of
function we already know. The solution to this problem is to define a
new function, \textsl{the error function}, specifically for using to
do the integral:
\begin{equation}
\mbox{erf}\,(x)=\frac{1}{\sqrt{\pi}}\int_{-x}^xe^{-y^2}dy=\frac{2}{\sqrt{\pi}}\int_0^xe^{-y^2}dy
\end{equation}
This is a so called \textsl{special function} which rougly means a
function we needed to define so that we could do an integral, or solve
a differential equation, we couldn't otherwise do or solve. Other
examples are Bessel functions and the elliptic integrals; there are
lots, especially coming from applied mathematics. Sometimes some
number theory functions, like Euler's totient function, are called
special functions. A lot of effort in the C19 was put into defining
special functions and finding efficient ways to numerically calculate
values; in those days, of course, these then went into big tables of
values; now all of that is done for us by the C \texttt{math} library
and it successors.

A graph of $\mbox{erf}\,(x)$ is shown in Fig.~\ref{fig_erf}. We can use it to work out Gau\ss{}ian probablities. Consider
\begin{equation}
\mbox{Prob}(x_1<x<x_2)=\int_{x_!}^{x_2} p(x)dx=\frac{1}{\sqrt{2\pi\sigma^2}}\int_{x_1}^{x_2} e^{-\frac{(x-\mu)^2}{2\sigma^2}}dx 
\end{equation}
Now let
\begin{equation}
z=\frac{x-\mu}{\sqrt{2}\sigma}
\end{equation}
so 
\begin{equation}
dz=\frac{dx}{\sqrt{2}\sigma}
\end{equation}
and when $x=x_1$ we have
\begin{equation}
z=z_1=\frac{x_1-\mu}{\sqrt{2}\sigma}
\end{equation}
and when $x=x_2$ we have
\begin{equation}
z=z_2=\frac{x_2-\mu}{\sqrt{2}\sigma}
\end{equation}
Substituting this all back into the integral
\begin{equation}
\mbox{Prob}(x_1<y<x_2)=\frac{1}{\sqrt{\pi}}\int_{z_1}^{z_2} e^{-z^2}dz=\frac{1}{\sqrt{\pi}}\int_{z_1}^{0} e^{-z^2}dz+\frac{1}{\sqrt{\pi}}\int_{0}^{z_2} e^{-z^2}dz
\end{equation}
Hence using the usual 
\begin{equation}
\int_a^bf(x)dx=-\int_b^a f(x)dx
\end{equation}
we have
\begin{equation}
\mbox{Prob}(x_1<x<x_2)=\frac{1}{2}[\mbox{erf}\,(z_2)-\mbox{erf}\,(z_1)]
\end{equation}

\subsection*{Example}

The loudness of songs at a concert are normally distributed\footnote{There is an inclindation to guess everything is normally distributed. In fact, this seems unlikely in this case, decibels are on a log scale}  with mean 75 dB and standard deviation $\sigma=10 dB$. What is the probability that the next song has loudness between 80 and 90 dB? Well
\begin{equation}
\mbox{Prob}(80<x<90)=\frac{1}{2}[\mbox{erf}\,(z_2)-\mbox{erf}\,(z_1)]
\end{equation}
where 
\begin{equation}
\sqrt{2}z_1=\frac{80-75}{10}=0.5
\end{equation}
and 
\begin{equation}
\sqrt{2}z_2=\frac{90-75}{10}=1.5
\end{equation}
Working out $\mbox{erf}\,(0.5/\sqrt{2})$ using your calculator or computer gives 0.38, whereas $\mbox{erf}\,(1.5/\sqrt{2})=0.87$ so
\begin{equation}
\mbox{Prob}(80<x<90)=0.2417
\end{equation}

\newpage

\ifind
\section*{Summary}
\else
\subsection*{1 Probability theory}
\fi

\begin{itemize}
\item A \textbf{sample space} is a set of point, they are the possible \textbf{outcomes} of a \textbf{trial}.
\item An \textbf{event} is a subset of a sample space.
\item A \textbf{probability} is a map from events to real numbers such that
  \begin{enumerate}
    \item $P(A)\ge 0$ for all events.
    \item $P(X)=1$
    \item If $A\cap B=\emptyset$ for two events $A$ and $B$ then 
      \begin{equation}
        P(A\cup B)=P(A)+P(B)
      \end{equation}
\end{enumerate}
\item A \textbf{probability mass function} is a map from points in the sample space to real numbers such that
  \begin{enumerate}
\item $p(x)\ge 0$ for all $x\in X$
\item $\sum_{x\in X} p(x)=1$
  \end{enumerate}
\item $P(A)=\sum_{x\in A}p(x)$
\item If all the points in a sample space have the same probability then
  \begin{equation}
P(A)=\frac{\mbox{number of points in }A}{\mbox{number of points in }X}=\frac{\#A}{\#{X}}
  \end{equation}
  where $\#(A)$ means the number of points in $A$.
  \item The \textbf{binomial coefficient}
\begin{equation}
\left(\begin{array}{c}n\\r\end{array}\right)=\frac{n!}{r!(n-r)!}
\end{equation}
counts the number of subsets of size $r$ in a set of $n$ objects and 
\begin{equation}
n!=n\times (n-1)\times (n-2)\times \ldots \times 2 \times 1
\end{equation}
\item The \textbf{partition function}
\begin{equation}
\left(\begin{array}{c} n\\n_1,n_2,\ldots,n_r\end{array}\right)=\frac{n!}{n_1!n_2!\ldots n_r!}
\end{equation}
where $n_1+n_2+\ldots+n_r=n$ counts the number of ways a set of $n$
objects can be split up into $r$ subgroups of sizes $n_1$, $n_2$ and
so on to $n_r$.

\end{itemize}



\ifind
\section*{Example question}
\else
\subsection*{5 binomial distribution - example question}
\fi
Show the variance of the binomial distribution is $npq$.

\noindent \textbf{solution}
This is a hard question. As ever we start with
\begin{equation}
1=Z=\sum_{r=0}^n \left(\begin{array}{c} n\\r\end{array}\right) p^rq^{n-r}
\end{equation}
and we take the second derivative
\begin{eqnarray}
\frac{d^2Z}{dp^2}&=&\sum_{r=0}^n \left(\begin{array}{c} n\\r\end{array}\right) r(r-1)p^{r-2}q^{n-r}-2\sum_{r=0}^n \left(\begin{array}{c} n\\r\end{array}\right) r(n-r)p^{r-1}q^{n-r-1}\cr&&+\sum_{r=0}^n \left(\begin{array}{c} n\\r\end{array}\right)(n-r)(n-r-1) p^rq^{n-r-2}
\end{eqnarray}
so
\begin{equation}
0=\sum_{r=0}^n\left[\frac{1}{p^2}(r^2-r)-\frac{2}{pq}(rn-r^2)+\frac{1}{q^2}(n^2-2nr-n+r^2+r)\right]\left(\begin{array}{c} n\\r\end{array}\right) p^rq^{n-r}
\end{equation}
or, using $\langle R\rangle = np$,
\begin{equation}
0=\frac{1}{p^2}\langle R^2\rangle - \frac{1}{p^2}np-\frac{2n}{pq}np+\frac{2}{pq}\langle R^2\rangle +(n^2-n)\frac{1}{q^2}-(2n-1)\frac{1}{q^2}np+\frac{1}{q^2}\langle R^2\rangle
\end{equation}
Putting this together we get
\begin{equation}
0=\left(\frac{1}{p}+\frac{1}{q}\right)^2\langle R^2\rangle - \frac{n}{p}-\frac{2n^2}{q}+ \frac{n^2-n}{q^2}-\frac{(2n^2-n)p}{q^2}
\end{equation}
Simplifying we get
\begin{equation}
0=\frac{\langle R^2\rangle}{p^2q^2}-\frac{n^2}{q^2}-\frac{n}{p}-\frac{n}{q}
\end{equation}
so
\begin{equation}
\langle R^2\rangle=n^2p^2+npq
\end{equation}
and finally
\begin{equation}
\sigma^2=\langle R^2\rangle-\langle R\rangle^2=npq
\end{equation}




\end{document}




